

\documentclass{article}
\usepackage[utf8]{inputenc}
\usepackage{setspace}
\usepackage{ mathrsfs }
\usepackage{graphicx}
\usepackage{amssymb} %maths
\usepackage{amsmath} %maths
\usepackage[margin=0.2in]{geometry}
\usepackage{graphicx}
\usepackage{ulem}
\setlength{\parindent}{0pt}
\setlength{\parskip}{10pt}
\usepackage{hyperref}
\usepackage[autostyle]{csquotes}

\usepackage{cancel}
\renewcommand{\i}{\textit}
\renewcommand{\b}{\textbf}
\newcommand{\q}{\enquote}
%\vskip1.0in





\begin{document}

{\setstretch{0.0}{


\b{Jagged} is a \q{stack of wheels}. The key is a stack of vectors of different length, and it's the sum of the first elements of all these vectors that does most of the work. This sum is called the \q{stack.} Then all those first elements are modified. The ciphertext symbol is added to each of them, mod $n$. Finally, all of the wheels are turned one unit. Since the wheels have different sizes, the changes caused by a particular ciphertext symbol are quickly diffused throughout the key. The essence of the program is in these three functions:

\begin{verbatim}

function twist!(f,a,n)
    for i in eachindex(f) f[i][1] = mod(f[i][1] + a,n) end
end

function roll!(f)
    for i in eachindex(f) f[i] = circshift(f[i],1) end
end

function encode(p,f,n)
    c = Int64[]
    for i in eachindex(p)
        push!(c, mod( stack(f) + p[i], n))
        twist!(f,c[i],n)
        roll!(f)  
    end
    c
end
\end{verbatim}



\end{document}
